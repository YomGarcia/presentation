\documentclass[8pt]{beamer}

\setbeamertemplate{background canvas}[vertical shading][bottom=cyan!10,top=blue!10]

\usetheme{Warsaw}
\usefonttheme[onlysmall]{structurebold}

% pour le fichiers .pdf
\usepackage{graphicx}
\usepackage{color}
% pour les fichiers .png
% \usepackage{pgf,pgfarrows}
% \usepackage{pgf,pgfarrows}
\usepackage{amsmath,amssymb}
\usepackage{textcomp}
\usepackage{Math_Notations}
\usepackage{multitoc}
\usepackage{mdwtab}
\setbeamercovered{dynamic}
\DeclareMathOperator*{\argmin}{argmin}

\title[OpenTURNS Developer Training]{OpenTURNS Developer Training: the platform overview}
\author[OpenTURNS Consortium, 2024]
{
  Trainer : R\'egis LEBRUN \\
  Airbus \\
  regis.lebrun@airbus.com
}

\date[19-22 March 2024]
{
  Developers training \\

  \begin{center}
    \includegraphics[height=2cm]{figures/logoOT.jpg}
  \end{center}
}

\subject{OpenTURNS Developers Training}

% \part<presentation>{Corps de presentation}


\begin{document}

\frame{\titlepage}

% necessaire pour la table des matieres
\part{Main part}

% table des matieres
\begin{frame}
  \Large
  \frametitle{Platform overview}
  \tableofcontents[part=1]
\end{frame}
%%%%%%%%%%%%%%%%%%%%%%%%% 
% The OpenTURNS package %
%%%%%%%%%%%%%%%%%%%%%%%%% 
\section[The story]{The story}
%%%%%%%%%%%%%%%% 
% Introduction %
%%%%%%%%%%%%%%%% 
\begin{frame}
  \frametitle{Introduction}
  \begin{block}{Objectives}
    The objectives of this course is to give a broad overview of the OpenTURNS project and the resulting platform. We will cover the following points:
    \begin{itemize}
    \item The history of the project,
    \item The global organization of the platform,
    \item The multi-layer view of the library,
    \item The several usages of the platform.
    \end{itemize}
  \end{block}
\end{frame}
%%%%%%%%%%% 
% History %
%%%%%%%%%%% 
\begin{frame}
  \frametitle{History}
  \begin{block}{2005-2024: 19 years of partnership}
    \begin{itemize}
    \item[2005] Conception
    \item[2007] First release the 10th of May. Python bindings.
    \item[2009] Multithreaded wrappers, polynomial chaos expansion.
    \item[2010] First windows port, parallelization.
    \item[2011] Sparse chaos implementation.
    \item[2012] v1.0, Stochastic processes
    \item[2013] Bayesian updating, matplotlib viewer
    \item[2014] Kriging, native windows support
    \item[2015] Vectorial kriging, HMat support
    \item[2016] Karhunen-Loeve process decomposition, NLopt bindings
    \item[2017] Canonical format low-rank tensor approximation, field functions
    \item[2018] Domains arithmetic, asymptotic Sobol' estimators, new simulation algorithms
    \item[2019] Calibration, new optimization algorithms, system events
    \item[2020] hmat AcaRandom compression, Spectra iterative SVD, examples gallery, XML/H5
    \item[2021] Karhunen-Loeve validation, new covariance models
    \item[2022] HSIC indices, iterative statistics, field chaos/sensitivity
    \end{itemize}
  \end{block}
\end{frame}
\section[The platform]{The platform}
%%%%%%%%%%%%%%%%%% 
% Global picture %
%%%%%%%%%%%%%%%%%% 
\begin{frame}
  \frametitle{The platform at a glance, developer's view}
  \centering \resizebox{!}{8cm}{\includegraphics{figures/GlobalPicture.pdf}}
\end{frame}
\begin{frame}
  \frametitle{The platform at a glance, developer's view}
  \begin{block}{The big parts}
    \begin{itemize}
    \item The \alert{core} of the OpenTURNS platform is a \alert{C++ library}, made of about 500 classes of various size. The library has a multi-layered architecture that is materialized by both the namespace hierarchy and the source tree.
    \item The \alert{main user interface} is the \alert{python module}, automatically generated from the C++ library using the wrapping software SWIG. It allows for a usage of OpenTURNS through python scripts of any level of complexity.
    \item The library relies on relatively \alert{few dependencies} and most of them are optional.
    \item A service of \alert{modules} is provided in order to extend the capabilities of the platform from the outside.
    \item Several \alert{GUIs} have already been built on top of the C++ library or the Python module.
    \end{itemize}
  \end{block}
\end{frame}
%%%%%%%%%%%%%%%%%%% 
% The C++ library %
%%%%%%%%%%%%%%%%%%% 
\begin{frame}
  \frametitle{The C++ library}
  \begin{block}{A multilayered library}
    The two main layers in the C++ library are the \alert{Base} layer and the \alert{Uncertainty} layer.
    \begin{itemize}
    \item \alert{Base} layer: it contains all the classes not related to the probabilistic concepts. It covers the elementary data types (vectors as Point, samples as Sample), the concept of models (Function), the linear algebra (Matrix, Tensor) and the general interest classes (memory management, resource management);
    \item \alert{Uncertainty} layer: it contains all the classes that deal with probabilistic concepts. It covers the probabilistic modelling (Distribution, RandomVector), the stochastic algorithms (MonteCarlo, FORM), the statistical estimation (DistributionFactory), the statistical testing (FittingTest)
    \end{itemize}
    A class in the Uncertainty layer can use any class in the Base or the Uncertainty layer. A class in the Base layer can ONLY USE classes in the Base layer.
  \end{block}
\end{frame}
\begin{frame}
  \frametitle{The C++ library}
  \begin{block}{A monolythic library?}
    The C++ library is provided as a unique object file (libOT.so) created using the libtool technology. As such, it is a monolythic library (of about 8Mo stripped), but internally it is made of numerous sub-libraries, one for each folder in the source tree. A future objective is to modularize this library in order to speed-up both the compilation time and the loading time.
  \end{block}
  \begin{block}{A parallel library}
    Some of the most time-consuming algorithms have been parallelized using the Thread Building Block technology (INTEL), a C++ library that allows for a parallelization of C++ code in a shared memory model. One of the objectives of OpenTURNS is the ability to execute external simulation softwares on large simulation models for a large amount of independent data sets. As such, OpenTURNS provides basic functionalities to distribute the executions of these simulations on a multiprocessors(cores) infrastructure using the low-level pthread technology or the TBBs.
  \end{block}
\end{frame}
%%%%%%%%%%%%%%%%%%%%% 
% The python module %
%%%%%%%%%%%%%%%%%%%%% 
\begin{frame}
  \frametitle{The python module}
  \begin{block}{Interfacing python and C++ libraries: SWIG}
    In order to provide a convenient interface to the user, the C++ library can be manipulated as a set of Python modules (18) that are organized through a hierarchy, on top of which stands the openturns module. The binding of the library is done almost automatically by SWIG (Simplified Wrapper Interface Generator) through a set of SWIG interface files.\\
    An additional significant work has been made in order to ease the interation between the OpenTURNS objects and the native Python objects.\\
    A unique feature of the Python interface is the ability to wrap a Python function into an OpenTURNS concept of mathematical function, namely the Function, using a specific class: the OpenTURNSPythonFunction. Using this mechanism, it is possible to adress virtually all the scenarios of coupling with an external simulation software.
  \end{block}
\end{frame}
%%%%%%%%%% 
% The OS %
%%%%%%%%%% 
\begin{frame}
  \frametitle{The target OSes}
  \begin{block}{A linux platform that works on windows}
    The historic platform of OpenTURNS is Linux. The first stage of developments have been made on 32 bits Intel Linux platforms (debian, mandriva), then the (minor) adaptations have been made in order to run on 64 bits Intel platforms as well.\\
    Currently, the platform works on the following platforms:
    \begin{itemize}
      \item Linux;
      \item macOS;
      \item Windows
    \end{itemize}
    The development platform remains Linux, the Windows version is obtained by cross compilation under Linux (using the MinGW tools). Native compilation is possible too.
  \end{block}
\end{frame}
%%%%%%%%%%%%%%%%%%%%%%%%%%%%%%%%%% 
% The development infrastructure %
%%%%%%%%%%%%%%%%%%%%%%%%%%%%%%%%%% 
\section[The development infrastructure]{The development infrastructure}
\begin{frame}
  \frametitle{The development infrastructure}
  \begin{block}{Compilation infrastructure}
    The present compilation infrastructure relies on CMake.
    In the past the autotools were used. 
    It covers:
    \begin{itemize}
    \item The detection and configuration aspects of the platform;
    \item The dependency management of the sources;
    \item The generation of parallel makefiles;
    \item The regression tests;
    \item The library packaging.
    \end{itemize}
    The use of CMake provides a way to compile the Windows version using Microsoft compilers, in order to easily reuse the C++ library in native Windows projects.
  \end{block}
\end{frame}
\begin{frame}
  \frametitle{The development infrastructure}
  \begin{block}{Versioning system}
    The versioning system used for the development of the whole platform is Git, the project is currently hosted by GitHub.
    \begin{itemize}
    \item \alert{Git} is a software versioning and a revision control system started by Linus Torvalds. It is used for the sources of the platform, the documentation, as well as for the development of modules.
    \item \alert{GitHub} is a web-based hosting service for version control using Git. GitHub allows hyperlinking information between a bug database, revision control and wiki content.
    \end{itemize}
  \end{block}
  \begin{block}{Repositories}
    Several repositories are used for the development of the platform, its documentation and its modules. This choice has been made for the following reasons:
    \begin{itemize}
    \item The time scale is not the same for these three activities;
    \item The teams are different partly in term of people, but mainly in term of expertise.
    \end{itemize}
  \end{block}
\end{frame}
%%%%%%%%%%%%%%%%%%%%%%%%%%%%%%%%%% 
% The development infrastructure %
%%%%%%%%%%%%%%%%%%%%%%%%%%%%%%%%%% 
\begin{frame}
  \frametitle{The development infrastructure}
  \begin{block}{Platform repository}
    The openturns Git repository is in charge of both the C++ library source code and the Python interface. It has the following layout, which is quite standard:
    \begin{itemize}
    \item A master branch that stores the source code of the upcoming version of the platform. The rule is to have only source code that pass with success all the tests embedded with both the library and the python module.
    \item Maintenance branches that contain stable versions.
    \item The development branches are part of contributors forks and integrated into master via pull requests.
    \item Several tags for each release candidates or official releases. 20 releases have been tagged so far in Git.
    \end{itemize}
    The usage of this infrastructure is described in the developer documentation, one of the several documents that come with the platform. In particular, a specific role is assigned to an \alert{integrator}, in charge of the merges from the different branches into master.
  \end{block}
\end{frame}
%%%%%%%%%%%%%%%%%%%%%%%%%%%%%%%%%% 
% The development infrastructure %
%%%%%%%%%%%%%%%%%%%%%%%%%%%%%%%%%% 
\begin{frame}
  \frametitle{The development infrastructure}
  \begin{block}{Continuous integration}
    OpenTURNS makes extensive use of online continuous integration providers to test the library throughout the development process. Currently the following services are used to test the library under several platforms:
    \begin{itemize}
    \item CircleCI: Linux and MinGW
    \item Travis: macOS
    \item Appveyor: MSVC
    \end{itemize}
  \end{block}
\end{frame}
%%%%%%%%%%%%%%%%%%%%%%%%%%%%%%%%%% 
% The development infrastructure %
%%%%%%%%%%%%%%%%%%%%%%%%%%%%%%%%%% 
\begin{frame}
  \frametitle{GitHub interface: the timeline}
  \centering \resizebox{!}{7cm}{\includegraphics{figures/Timeline.png}}
\end{frame}
%%%%%%%%%%%%%%%%%%%%%%%%%%%%%%%%%% 
% The development infrastructure %
%%%%%%%%%%%%%%%%%%%%%%%%%%%%%%%%%% 
\begin{frame}
  \frametitle{GitHub interface: the source navigator}
  \centering \resizebox{!}{7cm}{\includegraphics{figures/BrowseSource.png}}
\end{frame}
%%%%%%%%%%%%%%%%%%%%%%%%%%%%%%%%%% 
% The development infrastructure %
%%%%%%%%%%%%%%%%%%%%%%%%%%%%%%%%%% 
\begin{frame}
  \frametitle{GitHub interface: the bug tracking (1/2)}
  \centering \resizebox{!}{7cm}{\includegraphics{figures/Tickets1.png}}
\end{frame}
%%%%%%%%%%%%%%%%%%%%%%%%%%%%%%%%%% 
% The development infrastructure %
%%%%%%%%%%%%%%%%%%%%%%%%%%%%%%%%%% 
\begin{frame}
  \frametitle{GitHub interface: the bug tracking (2/2)}
  \centering \resizebox{!}{7cm}{\includegraphics{figures/Tickets2.png}}
\end{frame}
%%%%%%%%%%%%%%%%%%%%%%%%%%%%%%%%%% 
% The development infrastructure %
%%%%%%%%%%%%%%%%%%%%%%%%%%%%%%%%%% 
\section[Conclusion]{Conclusion}
\begin{frame}
  \Large
  \frametitle{Conclusion}
  \begin{itemize}
  \item A quite mature project: 17 full years of development
  \item A structured (rigid ;-)?) development process
  \item A working infrastructure to help the developer in his/her hard job!
  \end{itemize}
\end{frame}
\end{document}

