\documentclass[8pt]{beamer}

\setbeamertemplate{background canvas}[vertical shading][bottom=cyan!10,top=blue!10]

\usetheme{Warsaw}
\usefonttheme[onlysmall]{structurebold}

% pour le fichiers .pdf
\usepackage{graphicx}
\usepackage{color}
% pour les fichiers .png
% \usepackage{pgf,pgfarrows}
% \usepackage{pgf,pgfarrows}
\usepackage{amsmath,amssymb}
\usepackage{textcomp}
\usepackage{Math_Notations}
\usepackage{multitoc}
\usepackage{mdwtab}
\setbeamercovered{dynamic}
\DeclareMathOperator*{\argmin}{argmin}

\title[OpenTURNS Developer training]{OpenTURNS Developer training: first steps}
\author[OpenTURNS Consortium, 2024]
{
  Trainer : R\'egis LEBRUN\\
  Airbus \\
  regis.lebrun@airbus.com
}



\date[19-22 March 2024]
{
  Developers training \\

  \begin{center}
    \includegraphics[height=2cm]{logoOT.jpg}
  \end{center}
}

\subject{OpenTURNS Developers Training}

\begin{document}

\frame{\titlepage}

% necessaire pour la table des matieres
\part{Main part}

% table des matieres
\begin{frame}
  \frametitle{OpenTURNS: first steps}
  \tableofcontents[part=1]
\end{frame}

\section{Navigation in the source code}

%%%%%%%%%%%%%%%%%%%%%%%%%%%%%%%%% 
% Navigation in the source code %
%%%%%%%%%%%%%%%%%%%%%%%%%%%%%%%%% 
\begin{frame}
  \frametitle{Navigation in the source code}
  \begin{block}{The Uniform distribution}
    \begin{itemize}
    \item Locate the class within the library source code;
    \item Follow its inheritance graph in order to explore the Bridge pattern;
    \item Locate the associated regression test;
    \item Execute the test;
    \item Locate its SWIG interface file and its associated Python module;
    \item Execute the associated python test.
    \end{itemize}
  \end{block}
\end{frame}


\section{Library development}

%%%%%%%%%%%%%%%%%% 
% Global picture %
%%%%%%%%%%%%%%%%%% 
\begin{frame}
  \frametitle{Library development 1/5}
  \begin{block}{Projects}
    \begin{enumerate}
    \item~(*) \alert{\ttfamily InverseDistanceWeightingInterpolation} as a specialization of \alert{\ttfamily EvaluationImplementation} (see {\ttfamily lib/src/Base/Func}). Given a set of data $(x_i, y_i)_{i=1,\dots,N}$ in $\R^n\times\R^p$, the IDW interpolation is defined by:
      \begin{align}
        \forall x\in\R^n, u(x)=\left\{\begin{array}{ll}
                                        \dfrac{\sum_{i=1}^Nw_i(x)y_i}{\sum_{i=1}^Nw_i(x)} & \mbox{ if } \forall i, d(x, x_i) > 0\\
                                        y_i & \mbox{ if } \exists i, d(x, x_i)=0
                                      \end{array}
                                              \right.
      \end{align}
      where $w_i(x)=\dfrac{1}{d(x, x_i)^p}$ and $p>0$ a given smoothness parameter.

      The distance $d$ can be the Euclidean distance, the 1-norm or the sup norm.
      \item~(*) \alert{\ttfamily PiecewiseLinearDistribution} as a specialization of \alert{\ttfamily DistributionsImplementation} (see {\ttfamily lib/src/Uncertainty/Distribution}). Given a set of data $(x_i, y_i)_{i=1,\dots,N}$ in $\R^n\times\R^p$
      the PDF is linear over the data (with due renormalization).
    \end{enumerate}
  \end{block}
\end{frame}

\begin{frame}
  \frametitle{Library development 2/5}
  \begin{block}{Projects}
    \begin{enumerate}
      \setcounter{enumi}{7}
    \item~(*) \alert{\ttfamily TawnCopula} as a specialization of {\ttfamily ExtremeValueCopula} (see {\ttfamily lib/src/Uncertainty/Distribution}). This copula is defined by its Pickand function:
      \begin{align}
        \forall t\in[0,1], A(t)=(1-\psi_1)(1-t)+(1-\psi_2)t+\left[\left\{\psi_1t\right\}^{1/\theta}+\left\{\psi_2(1-t)\right\}^{1/\theta}\right]^\theta
      \end{align}
      where $0<\theta\leq 1$ and $0\leq\psi_1,\psi_2\leq 1$.
    \end{enumerate}
  \end{block}
\end{frame}

\begin{frame}
  \frametitle{Library development 3/5}
  \begin{block}{Projects}
    \begin{enumerate}
      \setcounter{enumi}{10}
    \item~(**) \alert{\ttfamily ArchiMaxCopula} as a specialization of {\ttfamily CopulaImplementation} (see {\ttfamily lib/src/Uncertainty/Distribution}). Given an Archimedean copula with generator $\psi$ and an extreme value copula with Pickand function $A$, an archimax copula $C$ is defined by:
      \begin{align}
        \forall (u,v)\in[0,1]^2, C(u,v)=\psi^{-1}\left(\min\left(\psi(0), [\psi(u)+\psi(v)]A\left(\dfrac{\psi(u)}{\psi(u)+\psi(v)}\right)\right)\right)
      \end{align}
      It becomes (***) if one wants to implement an efficient sampling algorithm.
    \end{enumerate}
  \end{block}
\end{frame}

\begin{frame}
  \frametitle{Library development 4/5}
  \begin{block}{Projects}
    \begin{enumerate}
      \setcounter{enumi}{12}
    \item~(***) Extend archimedian copulas from 2-d to $n$-d. Given a 2-d Archimedean copula with generator $\psi$, implement its $n$-d counterpart using:
      \begin{align}
        \forall (u_1,\dots,u_n)\in[0,1]^n, C(u_1,\dots,u_n)=\psi^{-1}\left(\psi(u_1)+\dots+\psi(u_n)\right)
      \end{align}
      The main difficulties are the architecture of this extension and the implementation of an efficient sampling algorithm.
    \end{enumerate}
  \end{block}
\end{frame}

\begin{frame}
  \frametitle{Library development 5/5}
  \begin{block}{Projects}
    \begin{enumerate}
      \setcounter{enumi}{15}
    \item~(*) Extend \alert{\ttfamily SolverImplementation} and \alert{\ttfamily Solver} to the resolution of systems of nonlinear equations and provide a generic implementation using the \alert{\ttfamily LeastSquaresProblem} class. The solutions $x^*$ of a nonlinear system of equations $f_1(x)=0,\dots,f_n(x)=0$ where $x=(x_1,\dots,x_n)$, if they exist, have to be found in the set of solutions of the following least-squares problem:
      \begin{align}
        x^*=\arg\min \sum_{j=1}^n f_j^2(x)
      \end{align}
      for which many solvers are available in OpenTURNS.
    \end{enumerate}
  \end{block}
\end{frame}

\section{Module development}

\begin{frame}
  \frametitle{Module development 1/2}
  \begin{block}{Projects}
    \begin{enumerate}
      \setcounter{enumi}{17}
    \item~(*) or (**) \alert{\ttfamily CloudMesher}: mesh generation over a cloud of points using kernel mixture, pca, rotation, then levelset mesher on an interval
    \end{enumerate}
  \end{block}
\end{frame}

\begin{frame}
  \frametitle{Module development 2/2}
  \begin{block}{Projects}
    \begin{enumerate}
      \setcounter{enumi}{21}
    \item~(**) \alert{\ttfamily CubaIntegration} as a specialization of {\ttfamily IntegrationAlgorithmImplementation} (see {\ttfamily lib/src/Base/algo}). This algorithm is obtained by interfacing the cuba C library. A possible name for the module is \alert{OTCuba}.
    \item~(**) \alert{\ttfamily HIntLibIntegration} as a specialization of {\ttfamily IntegrationAlgorithmImplementation} (see {\ttfamily lib/src/Base/algo}). This algorithm is obtained by interfacing the HIntLib C++ library, see \alert{https://github.com/JohannesBuchner/HIntLib}. A possible name for the module is \alert{OTHIntLib}.
    \end{enumerate}
  \end{block}
\end{frame}
\end{document}
